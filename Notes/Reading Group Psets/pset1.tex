\documentclass[11pt]{article}
\usepackage{amsmath, amsfonts, amssymb,amsthm}
\usepackage[includeheadfoot]{geometry} % For page dimensions
\usepackage{fancyhdr}
\usepackage{enumerate} % For custom lists
\usepackage{tikz-cd}
\usepackage{times}
\usepackage{hyperref}

\fancyhf{}
\lhead{GH Reading Group Pset1}
\rhead{Tighe McAsey}
\pagestyle{fancy}

% Page dimensions
\geometry{a4paper, margin=1in}

\theoremstyle{definition}
\newtheorem{defn}{Definition}
\newtheorem{thm}{Theorem}
\newtheorem{prop}{Proposition}
\newtheorem*{rmk}{Remark}
\newtheorem*{rmks}{Remarks}
\newtheorem{eg}{Example}
\newtheorem{exe}{Exercise}
\newtheorem{pb}{Problem}

% Commands:

\newcommand{\set}[1]{\{#1\}}
\newcommand{\abs}[1]{\lvert#1\rvert}
\newcommand{\norm}[1]{\lvert\lvert#1\rvert\rvert}
\newcommand{\gen}[1]{\left\langle #1 \right\rangle}
\newcommand{\tand}{\text{ and }}
\newcommand{\tor}{\text{ or }}
\newcommand{\falg}{F^{\text{alg}}}
\newcommand{\gal}{\text{Gal}}
\newcommand{\mor}{\text{Mor}}
\newcommand{\floor}[1]{\left\lfloor #1 \right\rfloor}
\newcommand{\coker}{\text{coker}}
\newcommand{\im}{\text{Im}}
\newcommand{\homo}{\text{Hom}}
\newcommand{\invlim}{\lim\limits_{\longleftarrow}}

\begin{document}
    \begin{pb}
        \textbf{(a)} \emph{Suppose \(f,g: \mathbb{C}^m \to \mathbb{C}\) are holomorphic, and for some open subset \(V \subset \mathbb{C}^n\) we have \(f\vert_V = g\vert_V\), show that \(f = g\).}
        
        \begin{proof}
            The Weierstrass theorem can be interpreted as non-zero holomorphic functions (in one variable) having isolated zeros, thus the result is already proved in \(1\) dimension and we can proceed by induction. define \(h = f-g\), and consider \(h\) as a functions of \((\mathbf{u},z)\) where \(\mathbf{u} \in \mathbb{C}^n\), we consider an arbitrary point \((\mathbf{u_0},z_0) \in \mathbb{C}^{n+1}\). fixing \((\mathbf{u'},z') \in V\) we have that \(h(\mathbf{u'},z) \equiv 0\) on \(\mathbb{C}\) so in particular \((\mathbf{u'},z_0) = 0\) by the Weierstrass theorem, there is some open neighborhood of this point \(U \subset V\), and by the same logic for each \((\mathbf{u},z') \in U\) we have \(h(\mathbf{u},z_0) = 0\). Thus fixing \(z = z_0\) we have that \(h(\mathbf{u},z_0)\) is locally zero on \(\mathbb{C}^n\), so by inductive hypothesis it is globaly zero and \(h(\mathbf{u_0},z_0) = 0\).
        \end{proof}

        \textbf{(b)} \emph{Prove the Maximum Modulus Principle.}
        
        \begin{proof}
            Consider any closed curve \(\gamma\) containing \(z_0\) with \(\gamma \subset U\). It follows that \(\abs{f(z_0)} \geq \abs{f(z)}\) on all of \(\gamma\). Then since \(\abs{f(z_0)}\) is constant it is holomorphic, so that
            \begin{align*}
                0 \leq \int_\gamma \abs{f(z_0)} - \abs{f(z)} = \int_\gamma \abs{f(z_0)} - \int_\gamma \abs{f(z)} \overset{\text{Cauchy's Thm.}}{=} \int_\gamma - \abs{f(z)} \leq 0
            \end{align*}
            For any point \(w \in U\), there is a closed curve \(\gamma \subset U\) passing through \(w\), if \(\abs{f(w)} < \abs{f(z)}\), then the above integral would be positive.
        \end{proof}

        \textbf{(c)} \emph{Prove elliptic regularity, i.e. complex differentiable implies smooth.}

        \begin{proof}
            It will suffice to show that holomorphic functions are equal to their power series, let \(z_0\) be in our domain, and \(\gamma\) be a circle of radius \(r\) about \(p\) in our domain (with \(z_0\) in the interior of \(gamma\)), then by Cauchy's integral formula:
            \begin{align*}
                f(z_0) &= \frac{1}{2\pi i}\int_{\gamma} \frac{f(z)}{z-z_0}dz = \frac{1}{2\pi i}\int_{\gamma} \frac{f(z)}{z-p} \frac{z-p}{z-p - (z_0 - p)}dz = \frac{1}{2\pi i}\int_{\gamma} \frac{f(z)}{z-p} \frac{1}{1 - \frac{z_0-p}{z-p}}dz \\
                &= \frac{1}{2\pi i}\int_{\gamma} \frac{f(z)}{z-p} \sum_0^\infty \left(\frac{z_0-p}{z-p}\right)^n = \frac{1}{2\pi i}\int_{\gamma}\sum_0^\infty \frac{f(z)}{(z-p)^{n+1}}(z_0-p)^n
            \end{align*}
            Since \(f\) is continuous and \(\gamma\) is compact, we have that \(\sup_\gamma\abs{f} = M\), and it is immediate that \(\abs{z_0 - p} < r = \abs{z - p}\) for all \(z \in \gamma\). In particular this implies that the sum of functions is abslutely convergent. Swapping the sum and integral gives us
            \begin{align*}
                f(z_0) = \frac{1}{2\pi i} \sum_0^\infty (z - p)^n \int_\gamma \frac{f(z)}{(z - p)^{n+1}}dz
            \end{align*}
        \end{proof}
        \begin{rmks}\; \newline
            \begin{itemize}
                \item \emph{In (a) the proof technique is familiar to me, it can be used to prove topological properties of varieties. e.g. cartesian products of infinite sets are dense in \(\mathbb{A}^n\).}
                \item \emph{(c) is just copying the methods of proof in GH, the algebra is explicated and uniform convergence, although obvious, is explained more carefully.}
            \end{itemize}
        \end{rmks}
    \end{pb}
    \begin{pb}
        \emph{Let \(f(z,w) = \sin(w^2) - z\), find the Weierstrass polynomial \(g\), such that \(f = gh\)}

        \begin{proof}
            Follow the process of the proof of the Weierstrass preparation theorem, in which case we find that \(b_1 = \sqrt{\arcsin(z)}, b_2 = -\sqrt{\arcsin(z)}\). This implies that \(g = w^2 - \arcsin(z)\), here we choose the branch of \(\arcsin\) with \(0 \mapsto 0\). Now we can once again take \(h\) as in the proof of the theorem, so that
            \begin{align*}
                h(0,0) = \frac{1}{2\pi i}\int_\gamma \frac{\sin(t^2) - 0}{t^2 - \arcsin(0)}\frac{1}{t}dt = \frac{1}{2\pi i}\int_\gamma \frac{\sin(t^2)}{t^3}dt \overset{\text{Laurent Series + res. Thm}}{=} 1 \neq 0
            \end{align*}
        \end{proof}

        \textbf{Remarks.}
        \begin{itemize}
            \item Away from \(\sin(w^2) = z\) we have \(h(z,w) = \frac{\sin(w^2) - z}{w^2 - \arcsin(z)}\), at \(\sin(w^2) = z\), we need to be more careful and compute \(h(z,w)\) from the definition \(h(z,w) = \frac{1}{2\pi i}\int_\gamma\frac{\sin(t^2) - z}{t^2 - \arcsin(z)}\frac{1}{t-w}dt\) as above.
            \item Near \(0\), we have that \(\arctan(z)\) satisfies the differential equation \(\frac{d}{dz}\phi(z) = \frac{1}{\sqrt{1 + z^2}}\), so we can define it as \(\int_0^z \frac{1}{\sqrt{1 + t^2}}dt\), this is clearly holomorphic near zero and \(\arcsin(z) = \arctan(\frac{z}{\sqrt{1 - z^2}})\) is also holomorphic near zero. Alternatively, we know that \(\arcsin\) exists and is holomorphic by the \emph{Holomorphic inverse function theorem}.
        \end{itemize}
    \end{pb}
    \begin{pb}
        \emph{Find \(\tau\), such that for \(\Lambda = \mathbb{Z}\oplus i\tau \mathbb{Z}\) we have \(C^\times/z\sim 2z \cong \mathbb{C}/\Lambda\) as complex manifolds.}

        \begin{proof}
            It is intuitive that the equivalence relation on \(\mathbb{C}^\times\) defines a torus, chasing the image through the complex logarithm will be sufficient to describe which torus. The following general observation is useful

            \begin{itemize}
                \item Let \(G\) be a group of \emph{biholomorphic} maps acting on a complex manifold \(M\) (totally discontinuous, free, etc.) suppose \((U_\alpha,\phi_\alpha)_\alpha\) is an atlas for \(M\) where the quotient map \(\pi\) is injective (it is straightforward that this always exists), then \((\pi(U_\alpha),\phi_\alpha\circ \pi\vert_{U_\alpha}^{-1})_\alpha\) is an atlas for \(M/G\) (in practice we can remove many sets from this atlas).
            \end{itemize}

            It is clear that multiplication by \(2^k\) is holomorphic (a diffeomorphism) on \(C^\times\), so it is easy to define charts on the quotient. I claim that \(\tau = \frac{2\pi}{\log 2}\). Since multiplication by a constant is holomorphic, it suffices to show that \(\mathbb{C}^\times/z\sim 2z \cong \mathbb{C}/\log2\mathbb{Z}\oplus 2\pi i \mathbb{Z}\). Taking \(z \mapsto \log\abs{z} + i\arg z\) it is obvious that this map defines a homeomorphism. We have charts on \(\mathbb{C}^\times\) being 
            \begin{align*}
                &U_1 = \left(\mathbb{C}^\times \setminus \set{\arg z = 0}\right) \cap \set{1 < \abs{z} < 2} \\ &U_2 = \left(\mathbb{C}^\times \setminus \set{\arg z = 0}\right) \cap \set{3/4 < \abs{z} < 3/2} \\ &U_3 = \left(\mathbb{C}^\times \setminus \set{\arg z = \pi}\right)\cap\set{1 < \abs{z} < 2} \\ &U_4 = \left(\mathbb{C}^\times \setminus\set{\arg z = \pi}\right)\cap \set{3/4 < \abs{z} < 3/2}
            \end{align*} and the inclusion into \(\mathbb{C}\) as \(\phi_i\). Now the above note gives a simple definition of charts on \(\mathbb{C}^\times/z \sim 2z\). We can also use the standard charts on the Torus.

            Now the charts on our quotient manifold, as well as the complex torus given by \(\tau\) are nicely symmetric, as such we can check on a single chart. Locally on \(U_1, V_1\) the maps are the complex logarithm away from its branch cut, and inversely the complex exponential, both are holomorphic.
        \end{proof}
        
        \textbf{Remarks.}
        \begin{itemize}
            \item Not all complex tori are biholomorphic
            \item Note that a biholomorphic map \(\mathbb{C}/\Lambda_1 \to \mathbb{C}/\Lambda_2\) induces a biholomorphic map of their universal covering spaces, i.e. \(\mathbb{C} \to \mathbb{C}\) with \(\Lambda_1 \to \Lambda_2\)
            \item All biholomorphic maps \(\mathbb{C} \to \mathbb{C}\) are affine
            \item Thus \(\Lambda_1 \tand \Lambda_2\) must be related via an affine transformation \(z \mapsto az + b\), shifting by \(b\) we require that \(\Lambda_1 = a \Lambda_2\) for a complex scalar.
        \end{itemize}
        
    \end{pb}
    \begin{pb} \emph{Show that \(\mathbb{CP}^n\) is a complex manifold, describe the transition functions.}
        \begin{proof}
            The charts are given by \(U_i = \set{z_i \neq 0}\) and \(\varphi_i: \mathbf{z} \mapsto (\frac{z_0}{z_i},\hdots,\hat{\frac{z_i}{z_i}},\hdots,\frac{z_n}{z_i})\). The coordinate change is given by \(\varphi_{ij}\) being multiplication by \(z_j/z_i\), note that although we cannot recover the original coordinates, this ratio is available from the local information.
        \end{proof}
        \emph{Do the same for the tautological line bundle}
        \begin{proof}
            Here the charts and coordinate changes are nearly the same, but we pair each \(U_i\) with a copy of \(\mathbb{C}^{n+1}\), and a trivialization \(\lambda\), representing \(\lambda \overline{z}\), thus the coordinate change on \(\lambda\) is given by multiplication by \(\frac{x_i}{x_j}\) as well.
        \end{proof}

        \textbf{Remark.}
        \begin{itemize}
            \item The sections of the tautological bundle can be identified with the space of \(-1\) degree homogenous polynomials in \(n+1\) variables, hence can be written as \(\mathcal{O}(-1)\). This also refects that there are no global sections on the tautological line bundle.
        \end{itemize}
    \end{pb}
\end{document}