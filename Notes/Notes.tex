\documentclass[11pt]{article}
\usepackage{amsmath, amsfonts, amssymb,amsthm}
\usepackage[includeheadfoot]{geometry} % For page dimensions
\usepackage{fancyhdr}
\usepackage{enumerate} % For custom lists
\usepackage{tikz-cd}

\fancyhf{}
\lhead{Notes}
\rhead{Tighe McAsey}
\pagestyle{fancy}

% Page dimensions
\geometry{a4paper, margin=1in}

\theoremstyle{definition}
\newtheorem{pb}{}

% Commands:

\newcommand{\set}[1]{\{#1\}}
\newcommand{\abs}[1]{\lvert#1\rvert}
\newcommand{\norm}[1]{\lvert\lvert#1\rvert\rvert}
\newcommand{\gen}[1]{\left\langle #1 \right\rangle}
\newcommand{\tand}{\text{ and }}
\newcommand{\tor}{\text{ or }}
\newcommand{\falg}{F^{\text{alg}}}
\newcommand{\gal}{\text{Gal}}
\newcommand{\mor}{\text{Mor}}
\newcommand{\floor}[1]{\left\lfloor #1 \right\rfloor}
\newcommand{\im}{\text{Im}}
\title{Notes for 2025}

\begin{document}
    \maketitle

    \section{Algebraic Topology}
    
    \emph{exercise - } Show that \(\mathbf{h} = (h_i)\), \(h_i: C_i \to C_{i+1}'\), then \(h_{i-1}\circ d_i + d_{i+1}' \circ h_i\) is a chain map.

    \emph{proof - } We need to check that \(d_i' \circ f_i = f_{i-1} \circ d_i\), in other words we need to show
    \begin{align*}
        d_i' \circ (h_{i-1}\circ d_i + d_{i+1}' \circ h_i) = (h_{i-2}\circ d_{i-1} + d_{i}' \circ h_{i-1})\circ d_i
    \end{align*}
    Since we are in a module, we can distribute \(d_i'\) on the left hand side, rewriting the condition as
    \begin{align*}
        d_i' \circ h_{i-1}\circ d_i + d_i' \circ d_{i+1}' \circ h_i = h_{i-2}\circ d_{i-1}\circ d_i + d_{i}' \circ h_{i-1}\circ d_i
    \end{align*}
    So it will suffice to show that
    \begin{align*}
        d_i' \circ d_{i+1}' \circ h_i = h_{i-2}\circ d_{i-1}\circ d_i
    \end{align*}
    But this is trivial since "\(d^2 = 0\)"
    % This follows from our assumption of commutativity of \(h_i\)-s with \(d_j\)-s, i.e. the equivalence is given by the commutativity of the following diagram
    % \begin{equation*}
    %     \begin{tikzcd}
    %         C_{i+1}\arrow[r] & C_{i} \arrow[r,"d_i"]\arrow[ld,"h_i"] &  C_{i-1} \arrow[r,"d_{i-1}"] &  C_{i-2} \arrow[ld,"h_{i-2}"]\\
    %         C_{i+1}' \arrow[r,"d_{i+1}'"] & C_i' \arrow[r,"d_i"] & C_{i-1}' \arrow[r] & C_{i-2}'
    %         \end{tikzcd}
    % \end{equation*}

    \emph{exercise - } A homotopy of chains is an equivalence relation

    \emph{proof - } We will prove the following items: reflexivity, symmetry, transitivity
    \begin{itemize}
        \item To see that \(f \sim f\), take each to be the zero map, \(h_i = 0, \forall i\). In this case the diagram with maps given by \(\mathbf{h}\) is immediate.
        \item Suppose that \(f \sim g\), then we have some \(\mathbf{h}\), such that \(g_i -f_i = h_{i-1}\circ d_i + d_{i+1}' \circ h_i\) for each \(i\). Since \(h_i\) are morphisms of modules, so are \(-h_i\), so in particular we have \(\mathbf{-h} := (-h_i)_i\), so that 
        \begin{align*}
            f_i - g_i &= -(g_i - f_i) = -(h_{i-1}\circ d_i + d_{i+1}' \circ h_i) = -h_{i-1}\circ d_i + -d_{i+1}' \circ h_i \\
            &= -h_{i-1}\circ d_i + d_{i+1}' \circ -h_i
        \end{align*}
        The last line follows since \(d_{i+1}'\) is linear.
        \item Suppose that \(f \sim g \sim r\), and let \(\mathbf{h, k}\) be respective witnesses of these homotopies. Then we have
        \begin{align*}
                &r_i - g_i = k_{i-1}\circ d_i + d_{i+1}'\circ k_i \\
                &g_i - f_i = h_{i-1}\circ d_i + d_{i+1}'\circ h_i
        \end{align*}
        This furnishes
        \begin{align*}
            &r_i - f_i = k_{i-1}\circ d_i + d_{i+1}'\circ k_i + h_{i-1}\circ d_i + d_{i+1}'\circ h_i
                       = (k_{i-1} + h_{i-1})\circ d_i + d_{i+1}'\circ(h_i + k_i)
        \end{align*}
        So we have the homotopy \(r \sim f\) via \(\mathbf{h} + \mathbf{k}\), we are done since we already proved symmetry.
    \end{itemize}

    \emph{exercise - } Represent diagrammatically what a homotopy of chain maps means

    \emph{proof - } Here we want \(\mathbf{h}\), so that
    \begin{align*}
        \cdots TODO \cdots
    \end{align*}

    \emph{exercise - } Show that the following two chain complexes are homotopic:
    \begin{equation*}
        \begin{tikzcd}
            0 \arrow[r] & \mathbf{Z} \oplus \mathbf{Z} \arrow[r,"{(\cdot , 2\cdot)}"] & \mathbf{Z} \oplus \mathbf{Z} \arrow[r] & 0 \oplus \mathbf{Z}/(2) \arrow[r] & 0 \\
            0 \arrow[r] & \mathbf{Z} \arrow[r,"2\cdot"] & \mathbf{Z} \arrow[r] & \mathbf{Z}/(2) \arrow[r] & 0
        \end{tikzcd}
    \end{equation*}
    
    \emph{proof - } Let each \(f_i\) be the projection of the second coordinate, and each \(g_i\) the inclusion into the second coordinate. In thise case we have
    \begin{align*}
        \mathbf{1}_{C'} - \mathbf{fg} = \mathbf{0}
    \end{align*}
    and hence \(\mathbf{h} = (0)_i\) witnesses the homotopy. The slightly harder case is the other direction. Define \(h_2 = h_0 = h_{-1} = 0\), and \(h_1: (m,n) \mapsto m\). By definition of \(\mathbf{f}, \mathbf{g}\), we have \(1_{C,i} - g_if_i: (m,n) \mapsto (m,0)\), so we just need to check that our given \(\mathbf{h}\) satisfies this.
    \begin{align*}
        &(d_3h_2 + h_1d_2)(m,n) = 0 + h_1(m,2n) = (m,0) \\
        &(d_2h_1 + h_0d_1)(m,n) = d_2(m,0) + 0 = (m,0) \\
        &(d_1h_0 + h_{-1}d_0)(0,n) = 0 + 0 = (0,0)
    \end{align*}
    This verifies the homotopy.

    \textbf{CAUTION!! - } The connection (this is what we call the diagonal maps often denoted \(\mathbf{h}\)) diagram need to not commute.

    \textbf{Theorem - } If chain complexes \(C,C'\) are homotopic, then they have the same Homology Modules.

    \emph{proof - } Recall that
    \begin{align*}
        H^i := \frac{\ker(d_i)}{\im(d_{i+1})}
    \end{align*}
    Now let the homotopy be given by \(\mathbf{f}:C \to C', \mathbf{g}:C' \to C\), there is a natural induced map of \(f_i\) on \(H^i\), given by restricting \(f_i\) to \(\im(d_{i+1})\), then taking the unique map from the quotient by the kernel of \(d_i\), which exists and is unique by the first isomorphism theorem. Calling this induced map \(f_{i,*}\), we need to check that
    \(f_{i,*}\) maps into \(H_i'\).

    \section{Commutative Algebra}

    \textbf{Hilbert's Basis Theorem - } if \(R\) is Noetherian, then \(R[X]\) is Noetherian
    
    \emph{proof.} Fix an ideal \(J\) of \(R[X]\), then \(J \cap \set{\deg = 0} \subset R\) is finitely generated. It is easy to verify by induction that
    \(J \cap \set{\deg \leq m}\) is finitely generated for any \(m\). Let \[I = \set{b \vert f \in J \tand f = bx^n + a_{n-1}x^{n-1} + \cdots + a_0}\]
    \(I \subset R\) is an ideal, thus is finitely generated, \(I = (b_i)_1^n\). Choose for each \(b_i\) some \(f_i\) where \(b_i\) is the leading coefficient.
    Then letting \(m = \max\set{\deg_{1\leq i\leq n} f_i}\)it is easy to verify using polynomial division and induction that
    \begin{align*}
        J = (f_i)_1^n + J \cap \set{\deg \leq m}
    \end{align*}
    This proves \(J\) is finitely generated. \qed

    \emph{It is easy to understand how someone could come up with this proof when we do it by first noticing that \(J \cap \set{\deg \leq m}\) is always finitely generated.
    This motivates the clever choice of ideal of leading coefficients since we want to reduce the degree.}

    \emph{exercise - } Prove that if \(R\) is Noetherian, and \(I \subset R\) is an ideal, then among the primes of \(R\) containing \(I\) there are only finitely many that are minimal with respect to inclusion.

    \emph{proof - } Assume not, then let 
    \[X = \set{I \subset R \mid \text{there are infinitely many minimal prime ideals containing } I}\]
    any chain in \(X\) has a maximal element by the Noetherian assumption, so Zorn's lemma furnishes a maximal element \(J\). \(J \in X\) implies that \(J\) is not prime, since then it would be the unique minimal ideal, hence there exist \(f,g \not \in J\) such that \(fg \in J\), it follows that for any minimal prime ideal \(P\) containing \(J\), we have \(fg \in P \implies f \in P \tor g \in P\), so that \(P \in J + (f) \tor J + (g)\), hence one of the two must be contained in infinitely many minimal prime ideals contradicting minimality. \qed

    \emph{exercise - }  Let \(M'\) be a submodule of \(M\). Show that \(M\) is Noetherian
    iff both \(M'\) and \(M / M'\) are Noetherian.

    \emph{proof - } If \(M\) is Noetherian, then so is \(M/M'\), since the image of the generators are generators. \(M'\) being Noetherian is just by passing chains in \(M'\) to the same chains in \(M\). Now suppose the converse, now let \(N_1 \subset N_2 \subset \cdots\) be a chain of submodules of \(M\), then there is some \(k'\), such that for all \(k \geq k'\) we have \(N_{k+1}/M' = N_k/M'\) and \(N_{k+1}\cap M' = N_k\cap M'\). Then let \(x \in N_{k+1}\), then there is some \(y \in N_k\) such that \(x-y \in M'\), hence
    \[x - y =m \in M' \implies m \in (N_{k+1} + N_k)\cap M' = N_{k+1} \cap M' = N_k \cap M'\]
    so \(x = y + m \in N_k\), hence \(N_{k+1} = N_k\). \qed

    \emph{exercise - } Let \(R = R_0 \oplus R_1 \oplus \cdots\) be graded, TFAE.
    \begin{enumerate}
        \item \(R\) is Noetherian
        \item \(R_0\) is Noetherian and \(R_1 \oplus R_2 \oplus \cdots\) is finitely generated
        \item \(R_0\) is Noetherian and \(R\) is finitely generated as a \(k\)-algebra
    \end{enumerate}

    \emph{proof - } 1 implies 2 is obvious. To show 3 implies 1, we have from Hilberts basis theorem \(R_0[x_1,\hdots,x_n]\) is Noetherian, since a finite \(R_0\)-algebra is a quotient of this free algebra, and quotients of Noetherian modules are Noetherian we are done. For 2 implies 3, let \(\set{f_i}_1^N\) be generators for \(R_1 \oplus R_2 \oplus \cdots\), then for \(f \in R\), \(f = \sum g_i f_i\), \(g_i\) in grading of \(\text{grad} f - \text{grad} f_i\) , recursing this process on \(g_i\), we eventually get "coefficients" with degree \(0\), so the module generators are also \(R_0\) algebra generators. \qed

    \emph{exercise - } Let \(R = k[x]\) show that every finitely generated \(R\)-module has a finite free resolution.

    \emph{proof - } Use the structure theorem
    \begin{equation*}
        \begin{tikzcd}
        0 \arrow[r] &k[x]^N \arrow[r] &k[x]^{r+N} \arrow[r]&  k[x]^r \bigoplus_1^N k[x]/p_i(x) \arrow[r] &0
        \end{tikzcd}
    \end{equation*}

    \emph{exercise - } Let \(R = k[x]/(x^n)\), compute a free resolution of the \(R\)-module \(R/(x^m)\), for any \(m \leq n\). Show that the only \(R\)-modules with finite
    free resolutions are the free modules. 

    \subsection{Tensor Products}

    \emph{Notes - }
    \begin{itemize}
        \item \((\cdot) \otimes N: \textbf{Mod}_\mathbf{R} \to \textbf{Mod}_\mathbf{R}\) is a right exact covarient functor.
    \end{itemize}

    \emph{exercise - } \(\mathbf{Z}/(10)\otimes \mathbf{Z}/(12) \simeq \mathbf{Z}/(2)\)

    \emph{proof.} Define \(\varphi: \mathbf{Z}/(10)\otimes \mathbf{Z}/(12) \to \mathbf{Z}/(2), \; a\otimes b \mapsto ab\). This is a homomorphism (mult is bilinear) and its clearly onto.
    To see the kernel is trivial, suppose \(a \otimes b \mapsto 0\), then \(a \otimes b = 2 (k \otimes \ell)\) by bilinearity. Then
    \begin{align*}
        2 (k \otimes \ell) = 2 (k \otimes 25\ell) = 50 (k \otimes \ell) = 5 (10k \otimes \ell) = 5(0 \otimes \ell) = \mathbf{0}
    \end{align*}

    \subsection{Localizations}

    \emph{Definition - } \(S \subset R\) multiplicatively closed, then \(S^{-1}R\) has elements \(a/s\), where \(a_1/s_1 = a_2/s_2\) exactly when there exists some \(s \in S\), such that \(s(s_2a_1 - s_1a_2) = 0\)
    We equip \(S^{-1}R\) withh the usual operations for fractions. This comes with the natural embedding
    \begin{align*}
        R \hookrightarrow S^{-1}R
    \end{align*}

    \emph{exercise - } \(R \hookrightarrow S^{-1}R\) is injective if and only if \(S\) does not contain zero divisors.

    \emph{Proof - } Suppose that \(S\) contains zero divisor \(s\), then for some \(r \in R\), \(sr = 0\), hence \(r \mapsto r/1 = 0/1\) which is the image of zero. Conversely, assume \(S\) has no zero divisors, then for any \(s\),
    \[s(a1 - r1) = 0 \iff a1 = r1 \iff a = \rho\]

    \emph{Definition of Localization as a Universal Property -} \(S^{-1}R\) is initial among \(A\)-algebras \(B\), where \(S \hookrightarrow B^\times\) (Note an \(R\)-Algebra is just a ring containing \(R\) thats also an \(R\)-Module)
    \begin{equation*}
        \begin{tikzcd}
            R \arrow[r, hook] \arrow[dr]
            & S^{-1}R \arrow[d, dashed]
            & S \to B^\times\\
            & B
        \end{tikzcd}
    \end{equation*}

    \emph{Definition of Localizations of Modules - } satisfies the same diagram as localizations of \(R\) (where we swap \(M\) for \(R\) and \(B\) for some arbitrary \(R\)-module \(N\)).
    
    \begin{itemize}
        \item To show that a localization of an \(R\)-module exists we follow the exact same construction as for rings, but here we may relabel \(a \in R \leadsto m \in M\)
    \end{itemize}

    \emph{exercise - } Show that \(S^{-1}\left(\bigoplus_i M_i\right) = \bigoplus_i S^{-1}M_i\) but not necessarily for infinite direct products

    Use the universal property. The explicit details of the proof are obvious (here we have \(1/s: a \mapsto b\), such that \(sb = a\)). To show it does not hold for direct products,
    \begin{align*}
        (1,1/2,1/3,\hdots) \in \prod_1^\infty \mathbf{Q} \setminus \mathbf{Q}\prod_1^\infty \mathbf{Z}
    \end{align*}
    Since any element of \(\mathbf{Q}\prod_1^\infty \mathbf{Z}\) has a "clearable denomenator"

    \newpage
    \section{Category Theory}

    \emph{General notes and thoughts - }
    \begin{itemize}
        \item Contravarience is naturally present in a pullback.
        \item A functor is right exact when for 
        \[M' \to M \to M'' \to 0\]
        an exact sequence (i.e. \(M'\) maps into the kernel of \(M \to M''\), \(M\) maps onto \(M''\)), then the sequence of induced maps is exact
        \[F(M') \to F(M) \to F(M'') \to 0\]
    \end{itemize}

    \emph{Categorical definition of a product -} Here \(P\) is the product of \(X,Y\), note that \(P\) is determined up to unique isomorphism.

    \begin{equation*}
        \begin{tikzcd}[row sep=1cm]
            &A\arrow[ddl,"f_1"']\arrow[ddr,"f_2"]\arrow[d,dotted,"\bar{f}" description]\\
            &P\arrow[dl,"p_1"]\arrow[dr,"p_2"']\\
            X&&Y
          \end{tikzcd}
    \end{equation*}

    \emph{Jargon - } A (covarient) functor is called faithful when for all \(A,B\), \(\mor(A,B) \to \mor(F(A),F(B))\) is injective, and full if it is surjective.
    An example of a full subcategory (under the inclusion map) is finitely generated \(R\)-Modules as a subcategory of \(R\)-Modules.

    \emph{exercise - } Initial objects are universal. 
    
    \emph{proof.} Suppose that \(A, B\) are two initial objects. There exist unique maps between them, when com posed these are maps from the oobjects to themselves, any map from the object to itself must be the identity by uniqueness of maps into an initial object, hence isomorphic. Uniqueness of this isomorphism follows by definition of initial object. \emph{The proof to show final objects are universal is nearly identical}.

    \emph{example - } Initial and final objects in \textbf{Set}, \textbf{Ring}, \textbf{Top}

    \begin{itemize}
        \item In \textbf{Set} and \textbf{Top} singletons are final.
        \item In \textbf{Ring}, \(0\) is initial and final.
    \end{itemize}

    \emph{Definition of fibered product - } The following is the definition of the fibered product \((P,h,g) = X \times_Z Y\)
    \begin{equation*}
        \begin{tikzcd}
            Q
            \arrow[bend left]{drr}{q_2}
            \arrow[bend right,swap]{ddr}{q_1}
            \arrow[dashed]{dr}[description]{\exists ! f} & & \\
            & P \arrow{r}{p_2} \arrow{d}[swap]{p_1}
            & Y \arrow{d}{g} \\
            & X \arrow[swap]{r}{h}
            & Z
            \end{tikzcd}
    \end{equation*}
\end{document}