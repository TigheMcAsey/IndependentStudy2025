\documentclass[11pt]{article}
\usepackage{amsmath, amsfonts, amssymb,amsthm}
\usepackage[includeheadfoot]{geometry} % For page dimensions
\usepackage{fancyhdr}
\usepackage{enumerate} % For custom lists

\fancyhf{}
\lhead{Notes}
\rhead{Tighe McAsey}
\pagestyle{fancy}

% Page dimensions
\geometry{a4paper, margin=1in}

\theoremstyle{definition}
\newtheorem{pb}{}

% Commands:

\newcommand{\set}[1]{\{#1\}}
\newcommand{\abs}[1]{\lvert#1\rvert}
\newcommand{\norm}[1]{\lvert\lvert#1\rvert\rvert}
\newcommand{\gen}[1]{\left\langle #1 \right\rangle}
\newcommand{\tand}{\text{ and }}
\newcommand{\tor}{\text{ or }}
\newcommand{\falg}{F^{\text{alg}}}
\newcommand{\gal}{\text{Gal}}
\newcommand{\floor}[1]{\left\lfloor #1 \right\rfloor}
\title{Notes for 2025}

\begin{document}
    \maketitle

    \section{Commutative Algebra}

    \textbf{Hilbert's Basis Theorem - } if \(R\) is Noetherian, then \(R[X]\) is Noetherian
    
    \emph{proof.} Fix an ideal \(J\) of \(R[X]\), then \(J \cap \set{\deg = 0} \subset R\) is finitely generated. It is easy to verify by induction that
    \(J \cap \set{\deg \leq m}\) is finitely generated for any \(m\). Let \[I = \set{b \vert f \in J \tand f = bx^n + a_{n-1}x^{n-1} + \cdots + a_0}\]
    \(I \subset R\) is an ideal, thus is finitely generated, \(I = (b_i)_1^n\). Choose for each \(b_i\) some \(f_i\) where \(b_i\) is the leading coefficient.
    Then letting \(m = \max\set{\deg_{1\leq i\leq n} f_i}\)it is easy to verify using polynomial division and induction that
    \begin{align*}
        J = (f_i)_1^n + J \cap \set{\deg \leq m}
    \end{align*}
    This proves \(J\) is finitely generated. \qed

    \emph{It is easy to understand how someone could come up with this proof when we do it by first noticing that \(J \cap \set{\deg \leq m}\) is always finitely generated.
    This motivates the clever choice of ideal of leading coefficients since we want to reduce the degree.}

    \subsection{Tensor Products}

    \emph{exercise - } \(\mathbf{Z}/(10)\otimes \mathbf{Z}/(12) \simeq \mathbf{Z}/(2)\)

    \emph{proof.} Define \(\varphi: \mathbf{Z}/(10)\otimes \mathbf{Z}/(12) \to \mathbf{Z}/(2), \; a\otimes b \mapsto ab\). This is a homomorphism (mult is bilinear) and its clearly onto.
    To see the kernel is trivial, suppose \(a \otimes b \mapsto 0\), then \(a \otimes b = 2 (k \otimes \ell)\) by bilinearity. Then
    \begin{align*}
        2 (k \otimes \ell) = 2 (k \otimes 25\ell) = 50 (k \otimes \ell) = 5 (10k \otimes \ell) = 5(0 \otimes \ell) = \mathbf{0}
    \end{align*}


    \section{Category Theory}

    \emph{exercise - } Initial objects are universal. 
    
    \emph{proof.} Suppose that \(A, B\) are two initial objects. There exist unique maps between them, when com posed these are maps from the oobjects to themselves, any map from the object to itself must be the identity by uniqueness of maps into an initial object, hence isomorphic. Uniqueness of this isomorphism follows by definition of initial object. \emph{The proof to show final objects are universal is nearly identical}.
\end{document}